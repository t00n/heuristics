\documentclass[a4paper,10pt]{article}
\usepackage[american]{babel}
\usepackage[utf8x]{inputenc}
\usepackage[left=2.5cm,top=2cm,right=2.5cm,nohead,nofoot]{geometry}
\usepackage{url}
\usepackage{listings}
\usepackage{graphicx}
\usepackage{subfigure}
\usepackage{float}

\title{Heuristic Optimisation\\Implementation 1}
\author{Antoine Carpentier}


\begin{document}

\maketitle

\section{Introduction}

In this project, we implement iterative improvement algorithms for the permutation flow-shop scheduling problem with weighted sum completion time objective. \\

In the first exercise, we implement 12 algorithms using combinations of 2 initialization methods, 2 pivoting rules and 3 neighbourhood types. The 2 initialization types are a random solution and a simplified RZ heuristic. To avoid bias in the first, we always set the random seed arbitrarely to 42. The 2 pivoting rules are first-improvement and best-improvement. The 3 neighbourhood types are the transpose, exchange and insert neighbourhoods. \\

In the second exercise, we implement a Variable Neighbourhood Descent. We use the random initialization method and the first-improvement pivoting rule and 2 orderings for the variable neighbourhood : transpose, exchange, insert and transpose, insert, exchange. \\

For both exercises, we compare the weighted sum completion time of the solution and the running time of the program in order to find the best algorithm to use for this problem. We also compare these metrics between the variable neighbourhoods and the single neighbourhoods. \\

Our implementation is in C++17. Just type `make' in the \textbf{src} directory. You will need a C compiler that supports C++17 (gcc 6.3 for example). Run the program with `./fssp -h' to display help. Results were extracted and computed with Python 3, Jupyter Notebook and Pandas.

\section{Exercise 1.1}

\subsection{Statistics}

Below are the running times and percentage deviation from the best known solution averaged for each algorithm and for each number of jobs.

\begin{figure}[H]
	\centering
	\caption{Statistics of exercise 1}
	\includegraphics[width=.8\textwidth]{img/exercise1_statistics.png}
\end{figure}


\subsection{Comparison}

To compare the several algorithms, we perform a Student t-test on the percentage deviation and the running time for each algorithm and for all instances with the null hypothesis that their average value are equal and with a confidence interval of 95\%. When the p-value is smaller than or equal to 0.05 we reject the hypothesis that the average values are equal. We then look at the sign of the t-statistic to know which average value is greater than the other one.

\begin{figure}[H]
	\centering
	\caption{Results of the Student t-test for exercise 1}
	\label{fig:exercise1_student}
	\subfigure[Comparison by initialization method (random/srz)]{
		\includegraphics[width=.25\textwidth]{img/exercise1_student_init.png}
	}
	\subfigure[Comparison by pivoting rule on the percentage deviation (best/first)]{
		\includegraphics[width=.25\textwidth]{img/exercise1_student_pivot_score.png}
	}
	\subfigure[Comparison by pivoting rule on the running time (best/first)]{
		\includegraphics[width=.25\textwidth]{img/exercise1_student_pivot_time.png}
	}
	\subfigure[Comparison by neighbourhood type on the percentage deviation (exchange/insert)]{
		\includegraphics[width=.25\textwidth]{img/exercise1_student_neigh_ei_score.png}
	}
	\subfigure[Comparison by neighbourhood type on the percentage deviation (exchange/transpose)]{
		\includegraphics[width=.25\textwidth]{img/exercise1_student_neigh_et_score.png}
	}
	\subfigure[Comparison by neighbourhood type on the percentage deviation (insert/transpose)]{
		\includegraphics[width=.25\textwidth]{img/exercise1_student_neigh_it_score.png}
	}
	\subfigure[Comparison by neighbourhood type on the running time (exchange/insert)]{
		\includegraphics[width=.25\textwidth]{img/exercise1_student_neigh_ei_time.png}
	}
	\subfigure[Comparison by neighbourhood type on the running time (exchange/transpose)]{
		\includegraphics[width=.25\textwidth]{img/exercise1_student_neigh_et_time.png}
	}
	\subfigure[Comparison by neighbourhood type on the running time (insert/transpose)]{
		\includegraphics[width=.25\textwidth]{img/exercise1_student_neigh_it_time.png}
	}
\end{figure} 

\paragraph{Which initial solution is preferable?}

We can see in \ref{fig:exercise1_student} (a) that for every algorithm the random and simplified RZ initialization methods have different average percentage deviation. In every comparison, the random method has a greater percentage deviation than the simplified RZ because its t-statistic is positive except for the case of the first-improvement pivoting rule with exchange neighbourhood where it is smaller. The simplified RZ initialization method seems to be the best.

\paragraph{Which pivoting rule generates better quality solutions and which is faster?}

In \ref{fig:exercise1_student} (b) we observe that we can reject the null hypothesis for every algorithm except the simplified RZ with exchange and transpose neighbourhoods. In the former cases, the t-statistic is positive so the first-improvement pivoting rule gives the best results. We cannot conclude anything for the the two latter cases. \\

Concerning the running time of each algorithm we can see in \ref{fig:exercise1_student} (c) that the best-improvement is often faster except for the transpose neighbourhood where we cannot reject the null hypothesis. This can be explained by the fact that the best-improvement algorithm will converge faster to a local optimum because it always jumps to the best solution in the neighbourhood.

\paragraph{Which neighborhood generates better quality solution and what computation time is required to reach local optima?}

In \ref{fig:exercise1_student} (d) (e) (f) we perform a Student t-test between each of the 3 neighbourhood types. We observe that we can reject the null hypothesis in every case except when the simplified RZ best-improvement exchange algorithm is compared with the insert algorithm. We can see in \ref{fig:exercise1_student} (d) that the exchange neighbourhood is better than the insert neighbourhood when used with a random initialization method and in \ref{fig:exercise1_student} (e) always better than the transpose neighbourhood. We conclude that the exchange neighbourhood might give the best results.

The last 3 tables of \ref{fig:exercise1_student} show the results for the running times. In \ref{fig:exercise1_student} (g) we see that we cannot reject the null hypothesis which means that the running times of the exchange and insert neighbourhood might be equal. In \ref{fig:exercise1_student} (h) (i) we see that we can reject the null hypothesis and that the transpose neighbourhood is always faster than the two others.

\section{Exercise 1.2}

\subsection{Statistics}

Below are the running times and percentage deviation from the best known solution averaged for each algorithm and for each number of jobs.

\begin{figure}[H]
	\centering
	\caption{Statistics of exercise 2}
	\includegraphics[width=.4\textwidth]{img/exercise2_statistics.png}
\end{figure}


The next table shows the percentage improvement of VND compared to single neighbourhoods. The percentage improvement is defined as the quotient of the percentage deviation of the neighbourhood in the rows on the percentage deviation of the neighbourhood in the columns.
\begin{figure}[H]
	\centering
	\caption{Percentage improvement of VND heuristics compared to single neighbourhoods}
	\includegraphics{img/exercise2_avg_percentage_improvement.png}
\end{figure}

\subsection{Comparison}

In the following table we show the results of a Student t-test between the percentage deviation of the two orderings of the VND and of the 3 single neighbourhoods. The initialization method is always random and the pivoting rule is first-improvement. We can reject the null hypothesis for every comparison. We see that the transpose-exchange-insert ordering of the VND is better than the insert and the transpose neighbourhoods and wore than the exchange neighbourhood. The results are similar for the transpose-insert-exchange VND. When we compare the two orderings of the VND we observe that the transpose-exchange-insert ordering is slightly better. \\

In regard of the fact that the exchange neighbourhood reaches better results than the two VND we ran, we can suppose that a VND ordering beginning by the exchange neighbourhood would be the best. Looking at the results of \ref{fig:exercise1_student} (f) we see that the insert neighbourhood obtains better results than the transpose neighbourhood. The best ordering could then be \textbf{exchange-insert-transpose}.

\begin{figure}[H]
	\centering
	\caption{Results of the Student t-test comparing percentage deviation of VND heuristics to single neighbourhoods}
	\includegraphics{img/exercise2_student_percentage_deviation.png}
\end{figure}

\end{document}

